\documentclass{paper}
\usepackage{}
\begin{document}
 \title{Heads Report}
Heads Report

The study of friction models is important to help the closed-loop camera stabilization system (HEADS) achieve its project specifications. During tests with Computed Torque + PID control, a stick-slip phenomenon has been observed as described in friction literature.

Summing up, this phenomenon is caused when the friction force is high enough to make the system stick at a different position then the reference one. Then the system moves, but go past the reference position (because of integral term in PID and lower friction torque during Slide regime) causing it to stick again at a different position. This behavior can be seen as an asymptotically stable limit-cycle.

Because of the torque lower than static friction and the integral term in PID control, a rising control signal can be seen, this will happen while the input torque is lower than the break-away friction (Pre-sliding regime). Literature references shows that the friction   

It occurs due to error caused by friction, which will make the integral part from PID accumulate the error, causing the system to slip until the torque is less or equal than static friction, causing it to stick.
    
In order to minimize its influence at the systems performance, some models have been studied and implemented in Matlab, such as Coulomb (static), Viscous, Integrated, Dahl, Stribeck and LuGre.

The main goal from this study is to allow a friction compensation during online operation. Therefore, friction parameters identification is required and a proper friction model (behavior close to the one seen empirically)

For now the main difficulty is to separate the torque caused by the friction from the control. Doing that allow us to have a better understanding of what characteristics the friction presents in Pre-Slide and Sliding regime.

\section{Friction Methodology}
According to Kermani,2007 

\section{Tests}

\textbf{2018-12-10}

Tests with coulomb Friction were made in order to better understand its influence in system dynamics. Due to its equation, it presents high frequency switch when the input torque is lower than $F_{static}$.

$ F_{coulomb} = sign(\dot{x})F_{static}$

Coulomb Friction and Viscous Friction model were implemented, but are not useful due to lack of realistic behavior.

The main problem occurs when the real system has stick-slip effect. Dahl and LuGre are some models that could present that behavior.

\end{document}